\documentclass{tmr}

\title{Editorial}

\author{Edward Z. Yang\email{ezyang@cs.stanford.edu}}

\begin{document}

This issue, we bring to you two articles which tie Haskell together with
other domains outside of the ordinary Haskell experience.  One combines
Haskell with machine learning; the other combines Haskell with computational
quantum chemistry.  These articles don't use the most sophisticated
type-level programming or Kan extensions; however, I do think they offer
a glimpse at the ways practitioners in other fields use Haskell.  I think
it's quite interesting to see what kinds of problems they care about
and what features of Haskell they lean on to get things done.  I hope you
agree!


\end{document}
